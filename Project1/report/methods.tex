\section{Trapped hard sphere Bose gas setup}
\label{sec:Trapped hard sphere Bose gas setup}
The trapped hard sphere Bose gas is a system of $N$ bosons in a trap with a hard core repulsion.
The trap we will use is either spherical (S) or elliptical (E) described by the potentials
\begin{equation}
\label{eq:extpotential}
V _{ext}(r) = 
{
	\left\{
		\begin{array}{lll}
			\frac{1}{2}m \omega _{ho}^2r^2 & (S)  \\
			\frac{1}{2}m[\omega _{ho}^2(x^2+y^2)+\omega _z^2z^2 & (E) 
		\end{array}
	\right.
}
\end{equation}
The trial wave function is given by
\begin{equation}
	\Psi _T(r) = \Psi _T(r_1, r_2, ..., r_N, \alpha , \beta ) = \left[\prod _{i=1}g(\alpha , \beta , r_i)\right]\left[\prod_{j<k}^{} f(a, |r_j-r_k|)\right],  
\end{equation}
where $g(\alpha , \beta , \boldsymbol{r_i})$ is the trial wave function for a single particle in the trap and $f(a, |\boldsymbol{r_j}-\boldsymbol{r_k}|)$ is the correlation wave function for the two-body potential 
\begin{equation}
g(\alpha, \beta , \boldsymbol{r_i}) = \exp(-\alpha (x_i^2 + y_i^2 + \beta z_i^2)), \quad f(a,|\boldsymbol{r_i}-\boldsymbol{r_j}|)=
{
	\left\{
		\begin{array}{lll}
			0 & \mbox{if } |\boldsymbol{r_i}-\boldsymbol{r_j}| \le a \\
			(1-\frac{a}{|\boldsymbol{r_i}-\boldsymbol{r_j}|} & \mbox{if } |\boldsymbol{r_i}-\boldsymbol{r_j}|>a. 
		\end{array}
	\right.
}
\end{equation}
The Hamiltonian operator $H$ for the system is given by 
\begin{equation}
\label{eq:Hamiltonian operator}
H = \sum_{i=1}^{N} [\frac{-\hbar^2}{2m} \nabla_i^2 + V _{ext}(r_i)] + \sum_{i<j}^{N} V _{int}(r_i,r_j).
\end{equation}
\
\section{Local energy} % (fold)
\label{sec:Local energy}
The local energy is given by 
\begin{equation}
\label{eq:local energy}
E_L(r) = \frac{1}{\Psi _T(r)}H\Psi _T(r),
\end{equation}
\subsection{Trial wave function without two-body potential}
\label{sec:Trial wave function without two-body potential}
The trial wave function without two-body potential is given by
\begin{equation}
\label{eq:trial wave function without two-body potential}
	\Psi_T(r) = \prod_{i}^{} \exp(-\alpha (x_i^2 + y_i^2 + \beta z_i^2)) .
\end{equation}
Using \autoref{eq:local energy} we find the local energy of a single particle in the trap to be 
\begin{gather*}
	\frac{1}{\exp(-\alpha (x^2+y^2+\beta z^2))}\left[\frac{-\hbar^2}{2m} \nabla^2 + V _{ext}(r)\right]	\exp(-\alpha (x^2+y^2+\beta z^2)). 
\end{gather*}
The laplace-operator $\Delta  = \grad^2$  acting on our trial wave function gives
\begin{gather*}
\Delta \exp(-\alpha (x^2+y^2+\beta z^2)) = \exp(-\alpha (x^2+y^2+\beta z^2)) \left(4\alpha^2 (x^2+y^2+\beta^2 z^2) - 4 \alpha -2\alpha \beta \right). 
\end{gather*}
With the spherical potential in \autoref{eq:extpotential} we have
\begin{gather*}
V _{ext}(r) \Psi _T(r) = \frac{1}{2}m \omega _{ho}^2(x^2+y^2+z^2) \exp(-\alpha (x^2+y^2+\beta z^2)).
\end{gather*}
The local energy for a single particle in the trap is then given by
\begin{gather*}
\frac{1}{exp(-\alpha (x^2+y^2+\beta z^2))}\left[\frac{-\hbar^2}{2m} \nabla^2 + V _{ext}(r)\right]	\exp(-\alpha (x^2+y^2+\beta z^2)) \\
= \frac{1}{2}m \omega _{ho}^2(x^2+y^2+z^2) - \frac{\hbar ^2}{4} \omega _{ho}^2 \left(4\alpha^2 (x^2+y^2+\beta^2 z^2) - 4 \alpha -2\alpha \beta \right).\\
\end{gather*}
For spherical traps we have $\beta = 1$ and for non-interacting bosons we have $\alpha =\frac{1}{2}a ^2 _{ho}$. Inserting 
these values we find the local energy to be 
\begin{gather*}
E_L(r) = \frac{1}{2}m \omega _{ho}^2(x^2+y^2+z^2) - \frac{\hbar ^2}{4} \omega _{ho}^2 \left(a _{ho}^4 (x^2+y^2+z^2) - 2 a^2 _{ho} -a^2 _{ho} \right) \\
= (x^2+y^2+z^2)\left(\frac{1}{2}m\omega _{ho}^2-\frac{\hbar ^2}{4}\omega _{ho}^2a^4 _{ho}\right)-3a^2 _{ho} \\
= r^2 \left(\frac{1}{2}m\omega _{ho}^2-\frac{\hbar ^2}{4}\omega _{ho}^2a^4 _{ho}\right)-3a^2 _{ho}.
\end{gather*}
We observe the local energy to be symmetric around the origin such that it is only dependent on the 
length $r$ of the position vector $\boldsymbol{r}$. 


For N particles in the same potential, the trial wave function is given by \autoref{eq:trial wave function without two-body potential} 
with $\beta =1$ and $\alpha =\frac{1}{2}a^2 _{ho}$. This may be written as
\begin{gather*}
\Psi_T(r) = \exp(-\alpha \sum_{i=1}^N (x_i^2 + y_i^2 + z_i^2)) .
\end{gather*}
The laplace-operator acting on the new wavefunction gives 
\begin{gather*}
	\sum_{i=1}^{N} \Delta _{i} \exp(-\alpha (x_i^2+y_i^2+z_i^2)) = \exp(-\alpha \sum_{i=1}^N (x_i^2 + y_i^2 + z_i^2)) \left(4\alpha^2 \sum_{i=1}^N 
	(x_i^2+y_i^2+z_i^2) - 4 \alpha N -2\alpha N \right).
\end{gather*}

\subsection{Trial wave function with two-body potential}
\label{sec:Trial wave function with two-body potential}
The trial wave function with two-body potential is given by
\begin{equation}
\label{eq:trial wave function with two-body potential}
	\Psi_T(r) = \left( \prod_{i}^{}g(\alpha ,\beta ,\boldsymbol{r_i})\right) \prod_{j<k}^{} f(a,|\boldsymbol{r_j-r_k}|).
\end{equation}
% section section name (end)
Defining the quantities 
\begin{gather*}
	r _{ij} = |\boldsymbol{r_i-r_j}|,\\ 
	ø(\boldsymbol{r_i})=\text{g}(\alpha ,\beta ,\boldsymbol{r_i})\quad \text{and}\\
	u(r _{ij})= \ln(f(r _{ij})),
\end{gather*}
we rewrite the trial wave function as
\begin{gather*}
\Psi_T(r) = \left( \prod_{i}^{}ø(\boldsymbol{r_i})\right) \exp\left(\sum_{j<k}^{} u(r _{jk})\right),
\end{gather*}
where 
\begin{gather*}
\sum_{i<j}^{N} X _{ij}\equiv \sum_{i=1}^{N} \sum_{j=i+1}^{N} X _{ij}. 
\end{gather*}
Using the product rule, we find the first derivative of particle $k$ to be
\begin{gather*}
\grad _{k} \Psi _{T}(\boldsymbol{r}) = \grad _{k}ø(\boldsymbol{r_k})\prod_{i\neq k}^{}ø(\boldsymbol{r_i})\exp\left(\sum_{j<m}^{} u(r _{jm})\right) + \prod_{i}^{}ø(\boldsymbol{r_i})\grad _k\exp\left(\sum_{j<m}^{} u(r _{jm})\right)\\
= \grad _{k}ø(\boldsymbol{r_k})\prod_{i\neq k}^{}ø(\boldsymbol{r_i})\exp\left(\sum_{j<m}^{} u(r _{jm})\right) + \prod_{i}^{}ø(\boldsymbol{r_i})\exp\left(\sum_{j<m}^{} u(r _{jm})\right)\grad_k \sum_{j<m}^{} u(r _{jm})\\
= \grad _{k}ø(\boldsymbol{r_k})\prod_{i\neq k}^{}ø(\boldsymbol{r_i})\exp\left(\sum_{j<m}^{} u(r _{jm})\right) + \prod_{i}^{}ø(\boldsymbol{r_i})\exp\left(\sum_{j<m}^{} u(r _{jm})\right) \sum_{l \neq  k} \boldsymbol{\nabla} _ku(r _{kl}). 
\end{gather*}
Applying the gradient again using the product rule, we find the laplace-operator acting on the trial wave function to be 
\begin{gather*}
\Delta _{k} \Psi _{T}(\boldsymbol{r}) = \Delta _{k}ø(\boldsymbol{r_k})\prod_{i\neq k}^{}ø(\boldsymbol{r_i})
\exp\left(\sum_{j<m}^{} u(r _{jm})\right) + \grad _{k}ø(\boldsymbol{r_k})\prod_{i\neq k}^{}ø(\boldsymbol{r_i})\exp\left(\sum_{j<m}^{} u(r _{jm})\right) \sum_{l \neq  k} \boldsymbol{\nabla} _ku(r _{kl})\\
+\boldsymbol{ \nabla }_kø(\boldsymbol{r_k}) \prod_{i \neq k}^{} ø(\boldsymbol{r_i}) 
\exp\left(\sum_{j<m}^{} u(r _{jm})\right) \sum_{l \neq  k} \boldsymbol{\nabla} _ku(r _{kl}) + \prod_{i}^{} 
ø(\boldsymbol{r_i}) \exp\left(\sum_{j<m}^{} u(r _{jm})\right) \sum_{l \neq k}^{} \boldsymbol{\nabla }_ku(r _{kl})\\ 
+\prod_{i}^{} ø(\boldsymbol{r_i})\exp\left(\sum_{j<m}^{} u(r _{jm}	)\right) 
\sum_{l \neq k}^{} \boldsymbol{\nabla }_ku(r _{kl}) 
\sum_{l \neq k} \boldsymbol{\nabla }_ku(r _{kl}).
\end{gather*}
Dividing this expression with the trial wave function, we get 
\begin{gather*}
\frac{\Delta _{k} \Psi _{T}(\boldsymbol{r})}{\Psi _{T}(\boldsymbol{r})}\\
=\Delta _{k}ø(\boldsymbol{r_k})\frac{1}{ø(\boldsymbol{r_k})} + \grad _{k}ø(\boldsymbol{r_k}) \frac{1}{ø(\boldsymbol{r_k})} \sum_{l \neq  k} \boldsymbol{\nabla} _ku(r _{kl})\\
+\boldsymbol{ \nabla }_kø(r_k)  \frac{1}{ø(\boldsymbol{r_k})} \sum_{l \neq  k} \boldsymbol{\nabla} _ku(r _{kl}) +  \sum_{l \neq k}^{} \boldsymbol{\nabla }_ku(r _{kl})
+\sum_{l \neq k}^{} \boldsymbol{\nabla }_ku(r _{kl}) \sum_{l \neq k} \boldsymbol{\nabla }_ku(r _{kl})\\
= \frac{\boldsymbol{\nabla} ^2ø(\boldsymbol{r_k})}{ø(\boldsymbol{r_k})}+ 2 \frac{\boldsymbol{\nabla }_k ø(\boldsymbol{r_k})}{ø(\boldsymbol{r}_k)} \sum_{l \neq k}^{} \boldsymbol{\nabla  }_ku(r _{kl})	
\end{gather*}

% section section name (end)	
